%
% File: chap_4.tex
% Author: Yuil Tripathee
%
\chapter{Systems Analysis and Design}

\section{Software Analysis}

\subsection{Class Diagram}

\textit{TODO: class diagram}

\subsection{Components Diagram}

\subsection{Sequence Diagram}

\subsubsection{Limitations}

\begin{description}
	\item[Unconventional implementation:] Coding syntax such as including SQL as part of modeling app sequences. This is not part of best practice as specified in the UML specifications. However, it serves us more effectively as the means of communication for different technical sub-teams (front end, backend and the deployment team). Our rationale behind this was to address the knowledge gap our current development team has. For the further updates, we intend to move up to the standard specification as knowledge gap is shortened and the backend team is ready to receive requirements inputs using the top level description.
	\item[Less flexibility to stack changes] For the current timeline, we chose the backend stack to adapt changes as quickly as possible. However, if the requirements changes results in change of technical stack (for example, we switch from HTTP REST \& SQL interface to GraphQL \&gRPC); our specification for interaction sequences will be obsolete. The SQL-inspired specification cannot address changes to other modern day NoSQL schema (such as document store, column store). Therefore, if we change the database structure the specifics in the sequence diagram regarding SQL interface should be changed.
\end{description}

\section{Systems Design}

% Define goals of project
% Systems Design -> decompose to smaller sub-systems
% Object Design -> define domain (object, interface, ...)

\subsection{Demonstration model}

\subsection{Full scale production model}

\clearpage