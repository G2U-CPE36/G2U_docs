%
% File: chap_7.tex
% Author: Yuil Tripathee
%
\chapter{Conclusion}

\section{Discussion}

Aligning to the lean startup model, we allowed ourselves to effectively adopt agile engineering workflow. We set the correct success criteria to obtain customer feedback as much as possible, rather than following our prejudice of how an e-commerce platform should be.

With the flexible team setup and feedback driven development, we gladly announce on this date, that we have achieved to develop and deploy an MVP (minimum viable product), that is good enough for closed testing within trusted circles.

\section{Future Work}

In the days to come, we intend to keep improvising our platform based on user feedback from various domains. In summary, we will rewards users to educate us on what they want and then deliver exactly that.

In summary, our technical road map for FY 2024/25 is presented as below. The goal of the technical team is to reward the participants (as much as possible) to retain and maintain the customer base.

\begin{description}
	\item[Q4 2024] Iterate the existing version of our application to adhere with security and business standards for European market.
	\item[Q1 2025] Improvise the existing version of our application to get market readiness to launch in Thai market.
	\item[Q2 2025] Integrate AI language model to improve localization and User Experience.
\end{description}

As for our business road-map, our goal is to sustain customer development process and keep getting credible feedback. As long as we have credible feedback, we are in the upper hand of business.

\begin{description}
	\item[Q1 2025] participate in marketing festivals, book event launches, and university open house as supporting sponsors.
	\item[Q2 2025] experiment and launch giveaway platform and include influencers to participate (those with good follower base on Instagram and Tiktok).
	\item[Q3 2025] launch lottery quest to maintain customer retention.
\end{description}

\section{Recommendation}

\subsection{Opportunities}

We are grateful to achieve these key distinctive learning outcomes, which is hard to achieve otherwise.

\begin{description}
	\item[Secrets to quality software] We learned (in a somehow hard way) that better code can make program better, but not a better software. We consider this to be the most important learning outcome. Until the software serves the intended need of user in the most effective way possible, software engineering processes should undergo necessary revisions by capturing user feedback with as much as actionable information as possible. % user feedback
	\item[Software Engineering Economics] A software engineering projects requires positive cash flow to survive and sustain. We explored and discovered several ways the project can be sustained in a long run with positive cash flow. This is crucial to us because we're going to explore the solution in a blue ocean environment. In such cases, the rules to deliver business values are not yet established, neither standardized. This requirement is quite relaxed in some environment though (like the companies who receive certain fees to complete outsourcing contracts). For us, optimum way to sustain our cash flow were scarce. However, lean startup came into rescue. It allows us to create a testing ground, while giving us space to test and validate new business models with optimum resource utilization.
	\item[Synergy] When we practice certain discipline for long time in quest of being a subject matter expert in a specialized domain (say server side development), it is quite easy to get tempted and fall into trap of being a possible single point of failure. This is a very risky move because in a short run, we might observe a fast progress. In a long run, lack of presence of that stakeholder alone can jeopardize the engineering operation. Therefore, we found a healthy practice to have each individual team members get a slight generalized knowhow of their few peers. If need be, any potential crisis due to the absence of that developer can be mitigated before the project faces an existential risk.
\end{description}

\subsection{Challenges}

If we had to do it over again, we learned that following challenges should be solved before the implementation phase begins:

\begin{description}
	\item[Line of communication] We have  when programming team members are unfamiliar to each other due to some reasons (such as unmatched learning curve). Front end and back end teams have different backgrounds that shape their idea of integrating two ends together. There are some workable solutions to try out such as add a personnel specializing in full stack development. Role of such person would be to develop interfaces in the back end part and sharing data directly to the front-end team as they like. In this case both parties can maintain consistency when there is changes to schema or business logic. However, this luxury might not be available all the times. Therefore, a common documentation such as shared OpenAPI specification \cite{2024_OpenAPI3.0} is found to be better alternatives. In order follow this approach, either team (front end or back end) has to put their desired specification in a shared document. Every time there is a breaking changes from either team, it has to go through this document first prior to code changes.
	
	\item[Retaining quality user's feedback] This is perpetually existing challenge in teams of all tiers (from startups to established enterprises). User testing is easier said than done. Because of the competitive market, developer teams have dilemma in terms of trade offs for confidentiality before their app release versus information they should publicly release. Businesses generally can not follow academic style survey methodologies due to various regulatory and technical reasons. This led us to conclude that we always need to explore ways to understand and track user appeal to our features.
\end{description}

\clearpage