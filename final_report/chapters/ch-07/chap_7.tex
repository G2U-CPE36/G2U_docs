%
% File: chap_7.tex
% Author: Yuil Tripathee
%
\chapter{Conclusion}

\section{Discussion}

\section{Future Work}

\section{Recommendation}

\subsection{Opportunities}

% Time to recall learning opportunities

\begin{description}
	\item[Secret to quality software] We learned (in a somehow hard way) that better code can make program better, but not a better software. % user feedback
\end{description}

\subsection{Challenges}

If we had to do it over again, we learned that following challenges should be solved before the implementation phase begins:

\begin{description}
	\item[Line of communication] We have  when programming team members are unfamiliar to each other due to some reasons (such as unmatched learning curve). Front end and back end teams have different backgrounds that shape their idea of integrating two ends together. There are some workable solutions to try out such as add a personnel specializing in full stack development. Role of such person would be to develop interfaces in the back end part and sharing data directly to the front-end team as they like. In this case both parties can maintain consistency when there is changes to schema or business logic. However, this luxury might not be available all the times. Therefore, a common documentation such as shared OpenAPI specification \cite{2024_OpenAPI3.0} is found to be better alternatives. In order follow this approach, either team (front end or back end) has to put their desired specification in a shared document. Every time there is a breaking changes from either team, it has to go through this document first prior to code changes.
	
	\item[Retaining quality user's feedback] This is perpetually existing challenge in teams of all tiers (from )
\end{description}

\clearpage